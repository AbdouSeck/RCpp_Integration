\documentclass[]{article}
\usepackage{lmodern}
\usepackage{amssymb,amsmath}
\usepackage{ifxetex,ifluatex}
\usepackage{fixltx2e} % provides \textsubscript
\ifnum 0\ifxetex 1\fi\ifluatex 1\fi=0 % if pdftex
  \usepackage[T1]{fontenc}
  \usepackage[utf8]{inputenc}
\else % if luatex or xelatex
  \ifxetex
    \usepackage{mathspec}
  \else
    \usepackage{fontspec}
  \fi
  \defaultfontfeatures{Ligatures=TeX,Scale=MatchLowercase}
\fi
% use upquote if available, for straight quotes in verbatim environments
\IfFileExists{upquote.sty}{\usepackage{upquote}}{}
% use microtype if available
\IfFileExists{microtype.sty}{%
\usepackage{microtype}
\UseMicrotypeSet[protrusion]{basicmath} % disable protrusion for tt fonts
}{}
\usepackage[margin=1in]{geometry}
\usepackage{hyperref}
\hypersetup{unicode=true,
            pdftitle={R and C++ Integration},
            pdfauthor={Abdou Seck},
            pdfborder={0 0 0},
            breaklinks=true}
\urlstyle{same}  % don't use monospace font for urls
\usepackage{color}
\usepackage{fancyvrb}
\newcommand{\VerbBar}{|}
\newcommand{\VERB}{\Verb[commandchars=\\\{\}]}
\DefineVerbatimEnvironment{Highlighting}{Verbatim}{commandchars=\\\{\}}
% Add ',fontsize=\small' for more characters per line
\usepackage{framed}
\definecolor{shadecolor}{RGB}{248,248,248}
\newenvironment{Shaded}{\begin{snugshade}}{\end{snugshade}}
\newcommand{\KeywordTok}[1]{\textcolor[rgb]{0.13,0.29,0.53}{\textbf{#1}}}
\newcommand{\DataTypeTok}[1]{\textcolor[rgb]{0.13,0.29,0.53}{#1}}
\newcommand{\DecValTok}[1]{\textcolor[rgb]{0.00,0.00,0.81}{#1}}
\newcommand{\BaseNTok}[1]{\textcolor[rgb]{0.00,0.00,0.81}{#1}}
\newcommand{\FloatTok}[1]{\textcolor[rgb]{0.00,0.00,0.81}{#1}}
\newcommand{\ConstantTok}[1]{\textcolor[rgb]{0.00,0.00,0.00}{#1}}
\newcommand{\CharTok}[1]{\textcolor[rgb]{0.31,0.60,0.02}{#1}}
\newcommand{\SpecialCharTok}[1]{\textcolor[rgb]{0.00,0.00,0.00}{#1}}
\newcommand{\StringTok}[1]{\textcolor[rgb]{0.31,0.60,0.02}{#1}}
\newcommand{\VerbatimStringTok}[1]{\textcolor[rgb]{0.31,0.60,0.02}{#1}}
\newcommand{\SpecialStringTok}[1]{\textcolor[rgb]{0.31,0.60,0.02}{#1}}
\newcommand{\ImportTok}[1]{#1}
\newcommand{\CommentTok}[1]{\textcolor[rgb]{0.56,0.35,0.01}{\textit{#1}}}
\newcommand{\DocumentationTok}[1]{\textcolor[rgb]{0.56,0.35,0.01}{\textbf{\textit{#1}}}}
\newcommand{\AnnotationTok}[1]{\textcolor[rgb]{0.56,0.35,0.01}{\textbf{\textit{#1}}}}
\newcommand{\CommentVarTok}[1]{\textcolor[rgb]{0.56,0.35,0.01}{\textbf{\textit{#1}}}}
\newcommand{\OtherTok}[1]{\textcolor[rgb]{0.56,0.35,0.01}{#1}}
\newcommand{\FunctionTok}[1]{\textcolor[rgb]{0.00,0.00,0.00}{#1}}
\newcommand{\VariableTok}[1]{\textcolor[rgb]{0.00,0.00,0.00}{#1}}
\newcommand{\ControlFlowTok}[1]{\textcolor[rgb]{0.13,0.29,0.53}{\textbf{#1}}}
\newcommand{\OperatorTok}[1]{\textcolor[rgb]{0.81,0.36,0.00}{\textbf{#1}}}
\newcommand{\BuiltInTok}[1]{#1}
\newcommand{\ExtensionTok}[1]{#1}
\newcommand{\PreprocessorTok}[1]{\textcolor[rgb]{0.56,0.35,0.01}{\textit{#1}}}
\newcommand{\AttributeTok}[1]{\textcolor[rgb]{0.77,0.63,0.00}{#1}}
\newcommand{\RegionMarkerTok}[1]{#1}
\newcommand{\InformationTok}[1]{\textcolor[rgb]{0.56,0.35,0.01}{\textbf{\textit{#1}}}}
\newcommand{\WarningTok}[1]{\textcolor[rgb]{0.56,0.35,0.01}{\textbf{\textit{#1}}}}
\newcommand{\AlertTok}[1]{\textcolor[rgb]{0.94,0.16,0.16}{#1}}
\newcommand{\ErrorTok}[1]{\textcolor[rgb]{0.64,0.00,0.00}{\textbf{#1}}}
\newcommand{\NormalTok}[1]{#1}
\usepackage{graphicx,grffile}
\makeatletter
\def\maxwidth{\ifdim\Gin@nat@width>\linewidth\linewidth\else\Gin@nat@width\fi}
\def\maxheight{\ifdim\Gin@nat@height>\textheight\textheight\else\Gin@nat@height\fi}
\makeatother
% Scale images if necessary, so that they will not overflow the page
% margins by default, and it is still possible to overwrite the defaults
% using explicit options in \includegraphics[width, height, ...]{}
\setkeys{Gin}{width=\maxwidth,height=\maxheight,keepaspectratio}
\IfFileExists{parskip.sty}{%
\usepackage{parskip}
}{% else
\setlength{\parindent}{0pt}
\setlength{\parskip}{6pt plus 2pt minus 1pt}
}
\setlength{\emergencystretch}{3em}  % prevent overfull lines
\providecommand{\tightlist}{%
  \setlength{\itemsep}{0pt}\setlength{\parskip}{0pt}}
\setcounter{secnumdepth}{0}
% Redefines (sub)paragraphs to behave more like sections
\ifx\paragraph\undefined\else
\let\oldparagraph\paragraph
\renewcommand{\paragraph}[1]{\oldparagraph{#1}\mbox{}}
\fi
\ifx\subparagraph\undefined\else
\let\oldsubparagraph\subparagraph
\renewcommand{\subparagraph}[1]{\oldsubparagraph{#1}\mbox{}}
\fi

%%% Use protect on footnotes to avoid problems with footnotes in titles
\let\rmarkdownfootnote\footnote%
\def\footnote{\protect\rmarkdownfootnote}

%%% Change title format to be more compact
\usepackage{titling}

% Create subtitle command for use in maketitle
\newcommand{\subtitle}[1]{
  \posttitle{
    \begin{center}\large#1\end{center}
    }
}

\setlength{\droptitle}{-2em}
  \title{R and C++ Integration}
  \pretitle{\vspace{\droptitle}\centering\huge}
  \posttitle{\par}
  \author{Abdou Seck}
  \preauthor{\centering\large\emph}
  \postauthor{\par}
  \predate{\centering\large\emph}
  \postdate{\par}
  \date{11/3/2017}


\begin{document}
\maketitle

\subsubsection{The R API}\label{the-r-api}

The R programming language offers an API to programmers through a set of
header files that ship with every R installation.

Essentially everything inside R is represented as a \texttt{SEXP} object
or \texttt{S-expression} in C++. By permitting exchange of such objects
between it and C++, R provides programmers with the ability to operate
directly on R objects.

There are two main functions that facilitate communication between C/C++
and R: \texttt{.C} and \texttt{.Call}. The former first appeared in an
earlier version of the R language and only supports pointers to basic C
types. The latter provides a richer interface and can operate directly
on \texttt{SEXP} objects.

The following are the main variants of \texttt{SEXP} objects:

\texttt{REALSXP}: numeric vector

\texttt{INTSXP}: integer vector

\texttt{LGLSXP}: logical vector

\texttt{STRSXP}: character vector

\texttt{VECSXP}: list

\texttt{CLOSXP}: function (closure)

\texttt{ENVSXP}: environment

\newpage

\subsubsection{Compiling and Linking}\label{compiling-and-linking}

You can manually compile your C++ code and link the resulting output to
your session. But it's easier to use the multitude of R packages that
make this process overly simple. Such packages include \texttt{inline}
and \texttt{Rcpp}. Both \texttt{inline} and \texttt{Rcpp} make it
extremely easy to compile and link C code to your current R session. We
will be using \texttt{Rcpp}.

\newpage

\subsubsection{Example}\label{example}

The following is a C++ function that can be called directly in R after
compilation and linking.

\begin{Shaded}
\begin{Highlighting}[]
\CommentTok{// In C/C++ ----------------------------------------}
\PreprocessorTok{#include }\ImportTok{<R.h>}
\PreprocessorTok{#include }\ImportTok{<Rinternals.h>}

\NormalTok{SEXP add(SEXP a, SEXP b) \{}
\NormalTok{  SEXP result = PROTECT(allocVector(REALSXP, }\DecValTok{1}\NormalTok{));}
\NormalTok{  REAL(result)[}\DecValTok{0}\NormalTok{] = asReal(a) + asReal(b);}
\NormalTok{  UNPROTECT(}\DecValTok{1}\NormalTok{);}

  \ControlFlowTok{return}\NormalTok{ result;}
\NormalTok{\}}
\end{Highlighting}
\end{Shaded}

The R code calling our function

\begin{Shaded}
\begin{Highlighting}[]
\CommentTok{# In R ----------------------------------------}
\NormalTok{add <-}\StringTok{ }\ControlFlowTok{function}\NormalTok{(a, b) \{}
  \KeywordTok{.Call}\NormalTok{(}\StringTok{"add"}\NormalTok{, a, b)}
\NormalTok{\}}
\end{Highlighting}
\end{Shaded}

\newpage

\subsubsection{Background}\label{background}

I mainly work with R when it comes to modeling, but any modeling task is
made possible by the heavy lifting of data wrangling. Often though,
wrangling can be too much of a task for R to deal with; especially when
there needs to be some scope conscious looping. By this, I mean any loop
wherein we have to worry about previous values. As a result, such
operations just cannot be easily \textbf{\emph{vectorized}}. This is a
situation when I feel that I need a \emph{lower} level and much more
perfomant language to do such tasks.

\newpage

\subsubsection{Sourcing the files}\label{sourcing-the-files}

Calling \texttt{source} allows R to run the R code inside the provided
file and save the result in the current namespace.
\texttt{Rcpp::sourceCpp} makes it easy to compile C++ code and link the
exported functions to the current R session.

\begin{Shaded}
\begin{Highlighting}[]
\KeywordTok{source}\NormalTok{(}\StringTok{'main_script.R'}\NormalTok{)}
\end{Highlighting}
\end{Shaded}

The script \texttt{main\_script.R} above contains the following code:

\begin{Shaded}
\begin{Highlighting}[]
\KeywordTok{lapply}\NormalTok{(}\KeywordTok{c}\NormalTok{(}\StringTok{"Rcpp"}\NormalTok{, }\StringTok{"RcppArmadillo"}\NormalTok{, }\StringTok{"microbenchmark"}\NormalTok{, }\StringTok{"inline"}\NormalTok{, ), }\ControlFlowTok{function}\NormalTok{(pkg) \{}
  \ControlFlowTok{if}\NormalTok{ (}\OperatorTok{!}\KeywordTok{require}\NormalTok{(pkg, }\DataTypeTok{character.only =}\NormalTok{ T)) \{}
    \KeywordTok{install.packages}\NormalTok{(pkg)}
\NormalTok{  \} }\ControlFlowTok{else}\NormalTok{ \{}
    \KeywordTok{sprintf}\NormalTok{(}\StringTok{"Package %s is already installed and loaded."}\NormalTok{, pkg)}
\NormalTok{  \}}
\NormalTok{\})}

\KeywordTok{setwd}\NormalTok{(}\StringTok{'~/Documents/Personal/Data_Analysis/R/RCpp_Integration/'}\NormalTok{)}
\CommentTok{# Get the R functions}
\KeywordTok{source}\NormalTok{(}\StringTok{'recursive_fib.R'}\NormalTok{)}
\KeywordTok{source}\NormalTok{(}\StringTok{'cached_fib.R'}\NormalTok{)}
\CommentTok{# Get the cpp functions}
\KeywordTok{sourceCpp}\NormalTok{(}\StringTok{'recursive_fib.cpp'}\NormalTok{)}
\KeywordTok{sourceCpp}\NormalTok{(}\StringTok{'cached_fib.cpp'}\NormalTok{)}
\end{Highlighting}
\end{Shaded}

\newpage

\subsubsection{Naive recursive implementation of Fibonacci
series}\label{naive-recursive-implementation-of-fibonacci-series}

\subsubsection{R}\label{r}

\begin{Shaded}
\begin{Highlighting}[]
\NormalTok{fibR <-}\StringTok{ }\ControlFlowTok{function}\NormalTok{(n) \{}
  \ControlFlowTok{if}\NormalTok{ (n }\OperatorTok{==}\StringTok{ }\DecValTok{0}\NormalTok{) }\KeywordTok{return}\NormalTok{(}\DecValTok{0}\NormalTok{)}
  \ControlFlowTok{if}\NormalTok{ (n }\OperatorTok{==}\StringTok{ }\DecValTok{1}\NormalTok{) }\KeywordTok{return}\NormalTok{(}\DecValTok{1}\NormalTok{)}
  \KeywordTok{return}\NormalTok{(}\KeywordTok{fibR}\NormalTok{(n}\OperatorTok{-}\DecValTok{1}\NormalTok{) }\OperatorTok{+}\StringTok{ }\KeywordTok{fibR}\NormalTok{(n}\OperatorTok{-}\DecValTok{2}\NormalTok{))}
\NormalTok{\}}
\end{Highlighting}
\end{Shaded}

\subsubsection{C++}\label{c}

\begin{Shaded}
\begin{Highlighting}[]
\CommentTok{// [[Rcpp::export]]}
\DataTypeTok{int}\NormalTok{ fibCpp(}\AttributeTok{const} \DataTypeTok{int}\NormalTok{ n) \{}
  \ControlFlowTok{if}\NormalTok{ (n == }\DecValTok{0}\NormalTok{) }\ControlFlowTok{return} \DecValTok{0}\NormalTok{;}
  \ControlFlowTok{if}\NormalTok{ (n == }\DecValTok{1}\NormalTok{) }\ControlFlowTok{return} \DecValTok{1}\NormalTok{;}
  \ControlFlowTok{return}\NormalTok{ fibCpp(n - }\DecValTok{1}\NormalTok{) + fibCpp(n - }\DecValTok{2}\NormalTok{);}
\NormalTok{\}}
\end{Highlighting}
\end{Shaded}

\subsubsection{Benchmarking}\label{benchmarking}

\begin{verbatim}
## Unit: microseconds
##        expr      min        lq       mean   median        uq       max
##    fibR(20) 7229.109 7786.0680 8481.31845 8153.650 9032.1840 15127.795
##  fibCpp(20)   46.681   47.6775   57.90319   55.309   58.9835  1035.562
##  neval
##    500
##    500
\end{verbatim}

\includegraphics{main_script_files/figure-latex/unnamed-chunk-7-1.pdf}

\newpage

\subsubsection{Cached recursive implementation of Fibonacci
series}\label{cached-recursive-implementation-of-fibonacci-series}

\subsubsection{R}\label{r-1}

\begin{Shaded}
\begin{Highlighting}[]
\NormalTok{cached_fibR <-}\StringTok{ }\KeywordTok{local}\NormalTok{(\{}
\NormalTok{  cache <-}\StringTok{ }\KeywordTok{c}\NormalTok{(}\DecValTok{1}\NormalTok{, }\DecValTok{1}\NormalTok{, }\KeywordTok{rep}\NormalTok{(}\OtherTok{NA}\NormalTok{, }\DecValTok{1000}\NormalTok{))}
\NormalTok{  f <-}\StringTok{ }\ControlFlowTok{function}\NormalTok{(n) \{}
    \ControlFlowTok{if}\NormalTok{ (n }\OperatorTok{<}\StringTok{ }\DecValTok{0}\NormalTok{) }\KeywordTok{return}\NormalTok{(}\OtherTok{NA}\NormalTok{)}
    \ControlFlowTok{if}\NormalTok{ (n }\OperatorTok{>}\StringTok{ }\KeywordTok{length}\NormalTok{(cache)) }\KeywordTok{stop}\NormalTok{(}\StringTok{"Cannot process for n greater than 1000."}\NormalTok{)}
    \ControlFlowTok{if}\NormalTok{ (n }\OperatorTok{==}\StringTok{ }\DecValTok{0}\NormalTok{) }\KeywordTok{return}\NormalTok{(}\DecValTok{0}\NormalTok{)}
    \ControlFlowTok{if}\NormalTok{ (n }\OperatorTok{==}\StringTok{ }\DecValTok{1}\NormalTok{) }\KeywordTok{return}\NormalTok{(}\DecValTok{1}\NormalTok{)}
    \ControlFlowTok{if}\NormalTok{ (}\OperatorTok{!}\KeywordTok{is.na}\NormalTok{(cache[n])) }\KeywordTok{return}\NormalTok{(cache[n])}
\NormalTok{    ans <-}\StringTok{ }\KeywordTok{f}\NormalTok{(n }\OperatorTok{-}\StringTok{ }\DecValTok{1}\NormalTok{) }\OperatorTok{+}\StringTok{ }\KeywordTok{f}\NormalTok{(n }\OperatorTok{-}\StringTok{ }\DecValTok{2}\NormalTok{)}
\NormalTok{    cache[n] <<-}\StringTok{ }\NormalTok{ans}
\NormalTok{    ans}
\NormalTok{  \}}
\NormalTok{\})}
\end{Highlighting}
\end{Shaded}

\newpage

\subsubsection{C++}\label{c-1}

\begin{Shaded}
\begin{Highlighting}[]
\PreprocessorTok{#include }\ImportTok{<Rcpp.h>}
\PreprocessorTok{#include }\ImportTok{<algorithm>}
\PreprocessorTok{#include }\ImportTok{<vector>}
\PreprocessorTok{#include }\ImportTok{<stdexcept>}
\PreprocessorTok{#include }\ImportTok{<cmath>}
\PreprocessorTok{#include }\ImportTok{<iostream>}
\KeywordTok{using} \KeywordTok{namespace}\NormalTok{ Rcpp;}

\KeywordTok{class}\NormalTok{ Fib \{}
  \KeywordTok{public}\NormalTok{:}
\NormalTok{    Fib(}\DataTypeTok{unsigned} \DataTypeTok{int}\NormalTok{ n = }\DecValTok{1000}\NormalTok{) \{}
\NormalTok{      cache.resize(n);}
      \BuiltInTok{std::}\NormalTok{fill(cache.begin(), cache.end(), NAN);}
\NormalTok{      cache[}\DecValTok{0}\NormalTok{] = }\FloatTok{0.0}\NormalTok{;}
\NormalTok{      cache[}\DecValTok{1}\NormalTok{] = }\FloatTok{1.0}\NormalTok{;}
\NormalTok{    \}}
    \DataTypeTok{double}\NormalTok{ cached_fibCpp(}\DataTypeTok{int}\NormalTok{ x) \{}
      \ControlFlowTok{if}\NormalTok{ (x < }\DecValTok{0}\NormalTok{) \{}
        \ControlFlowTok{return}\NormalTok{ (}\DataTypeTok{double}\NormalTok{) NAN;}
\NormalTok{      \}}
      \ControlFlowTok{if}\NormalTok{ (x >= (}\DataTypeTok{int}\NormalTok{) cache.size()) \{}
        \ControlFlowTok{throw} \BuiltInTok{std::}\NormalTok{range_error(}\StringTok{"x too large for implementation"}\NormalTok{);}
\NormalTok{      \}}
      \ControlFlowTok{if}\NormalTok{ (x < }\DecValTok{2}\NormalTok{) \{}
        \ControlFlowTok{return}\NormalTok{ x;}
\NormalTok{      \}}
      \ControlFlowTok{if}\NormalTok{ (! ::isnan(cache[x])) }\ControlFlowTok{return}\NormalTok{ cache[x];}
\NormalTok{      cache[x] = cached_fibCpp(x - }\DecValTok{1}\NormalTok{) + cached_fibCpp(x - }\DecValTok{2}\NormalTok{);}
      \ControlFlowTok{return}\NormalTok{ cache[x];}
      
\NormalTok{    \}}
  \KeywordTok{private}\NormalTok{:}
    \BuiltInTok{std::}\NormalTok{vector<}\DataTypeTok{double}\NormalTok{> cache;}
\NormalTok{\};}

\NormalTok{Fib f = Fib(}\DecValTok{2000}\NormalTok{);}

\CommentTok{// [[Rcpp::export]]}
\DataTypeTok{double}\NormalTok{ cached_fibCpp(}\AttributeTok{const} \DataTypeTok{int}\NormalTok{ a) \{}
  \ControlFlowTok{return}\NormalTok{ f.cached_fibCpp(a);}
\NormalTok{\}}
\end{Highlighting}
\end{Shaded}

\newpage

\subsubsection{Benchmarking}\label{benchmarking-1}

\begin{verbatim}
## Unit: microseconds
##                expr   min     lq     mean median     uq      max neval
##    cached_fibR(500) 1.382 1.4970 4.389010  1.558 1.6610 2613.308  1000
##  cached_fibCpp(500) 0.715 0.8725 2.056097  0.960 1.1115  803.504  1000
\end{verbatim}

\includegraphics{main_script_files/figure-latex/unnamed-chunk-10-1.pdf}

\newpage

\subsubsection{Iterative implementation of Fibonacci
series}\label{iterative-implementation-of-fibonacci-series}

\subsubsection{R}\label{r-2}

\begin{Shaded}
\begin{Highlighting}[]
\NormalTok{iter_fibR <-}\StringTok{ }\ControlFlowTok{function}\NormalTok{(x) \{}
  \ControlFlowTok{if}\NormalTok{ (x }\OperatorTok{<}\StringTok{ }\DecValTok{0}\NormalTok{) }\KeywordTok{return}\NormalTok{(}\OtherTok{NA}\NormalTok{)}
  \ControlFlowTok{if}\NormalTok{ (x }\OperatorTok{<}\StringTok{ }\DecValTok{2}\NormalTok{) }\KeywordTok{return}\NormalTok{(x)}
\NormalTok{  first <-}\StringTok{ }\DecValTok{0}
\NormalTok{  second <-}\StringTok{ }\DecValTok{1}
\NormalTok{  third <-}\StringTok{ }\DecValTok{0}
  \ControlFlowTok{for}\NormalTok{ (i }\ControlFlowTok{in} \KeywordTok{seq_len}\NormalTok{(x)) \{}
\NormalTok{    third <-}\StringTok{ }\NormalTok{first }\OperatorTok{+}\StringTok{ }\NormalTok{second}
\NormalTok{    first <-}\StringTok{ }\NormalTok{second}
\NormalTok{    second <-}\StringTok{ }\NormalTok{third}
\NormalTok{  \}}
\NormalTok{  first}
\NormalTok{\}}
\end{Highlighting}
\end{Shaded}

\subsubsection{C++}\label{c-2}

\begin{Shaded}
\begin{Highlighting}[]
\PreprocessorTok{#include }\ImportTok{<Rcpp.h>}
\PreprocessorTok{#include }\ImportTok{<cmath>}
\KeywordTok{using} \KeywordTok{namespace}\NormalTok{ Rcpp;}

\CommentTok{// [[Rcpp::export]]}
\DataTypeTok{double}\NormalTok{ iter_fibCpp(}\AttributeTok{const} \DataTypeTok{int}\NormalTok{ x) \{}
  \ControlFlowTok{if}\NormalTok{ (x < }\DecValTok{0}\NormalTok{) }\ControlFlowTok{return}\NormalTok{ NAN;}
  \ControlFlowTok{if}\NormalTok{ (x < }\DecValTok{2}\NormalTok{) }\ControlFlowTok{return}\NormalTok{ x;}
  \DataTypeTok{double}\NormalTok{ first = }\FloatTok{0.0}\NormalTok{;}
  \DataTypeTok{double}\NormalTok{ second = }\FloatTok{1.0}\NormalTok{;}
  \DataTypeTok{double}\NormalTok{ third = }\FloatTok{0.0}\NormalTok{;}
  \ControlFlowTok{for}\NormalTok{ (}\DataTypeTok{int}\NormalTok{ i = }\DecValTok{0}\NormalTok{; i < x; i++) \{}
\NormalTok{    third = first + second;}
\NormalTok{    first = second;}
\NormalTok{    second = third;}
\NormalTok{  \}}
  \ControlFlowTok{return}\NormalTok{ first;}
\NormalTok{\}}
\end{Highlighting}
\end{Shaded}

\newpage

\subsubsection{Benchmarking}\label{benchmarking-2}

\begin{verbatim}
## Unit: microseconds
##              expr    min      lq      mean  median     uq      max neval
##    iter_fibR(500) 29.268 31.4385 39.808805 32.6430 34.058 4862.766  1000
##  iter_fibCpp(500)  1.121  1.3725  2.844399  1.5305  1.755 1123.231  1000
\end{verbatim}

\includegraphics{main_script_files/figure-latex/unnamed-chunk-13-1.pdf}

\newpage

\subsubsection{Vector Autogression
Model}\label{vector-autogression-model}

A VAR is used to capture linear independencies among multiple time
series. In sum, it describes the evolution of a set of endogenous
variables over the same sample period \((t = 1, ..., T)\) as a linear
function of only their past values.

Considering the simplest case of a two-dimensional VAR of order one, we
can use the following notation:

\[Ax_{t-1} + u_t\]

At time \(t\) , the model is comprised of two endogenous variables
\(x_t = (x_{1t} , x_{2t})\) which are a function of their previous
values at \(t - 1\) via the coefficient matrix \(A\). When studying the
properties of VAR systems, simulation is a tool that is frequently used
to assess these models. And, for the simulations, we need to generate
suitable data. Due to the interdependence between the two coefficients
in the equation above, we can see that vectorization is not going to be
possible. Therefore, we will need to loop explicitly.

\subsubsection{R}\label{r-3}

\begin{Shaded}
\begin{Highlighting}[]
\NormalTok{A <-}\StringTok{ }\KeywordTok{matrix}\NormalTok{(}\KeywordTok{c}\NormalTok{(}\FloatTok{0.5}\NormalTok{, }\FloatTok{0.1}\NormalTok{, }\FloatTok{0.1}\NormalTok{, }\FloatTok{0.5}\NormalTok{), }\DataTypeTok{nrow=}\DecValTok{2}\NormalTok{)}
\NormalTok{u <-}\StringTok{ }\KeywordTok{matrix}\NormalTok{(}\KeywordTok{rnorm}\NormalTok{(}\DecValTok{10000}\NormalTok{), }\DataTypeTok{ncol=}\DecValTok{2}\NormalTok{)}

\NormalTok{Rsim <-}\StringTok{ }\ControlFlowTok{function}\NormalTok{(coeff, errors) \{}
\NormalTok{  simdata <-}\StringTok{ }\KeywordTok{matrix}\NormalTok{(}\DecValTok{0}\NormalTok{, }\KeywordTok{nrow}\NormalTok{(errors), }\KeywordTok{ncol}\NormalTok{(errors))}
  \ControlFlowTok{for}\NormalTok{ (row }\ControlFlowTok{in} \DecValTok{2}\OperatorTok{:}\KeywordTok{nrow}\NormalTok{(errors)) \{}
\NormalTok{    simdata[row, ] =}\StringTok{ }\NormalTok{coeff }\OperatorTok\StringTok{ }\NormalTok{simdata[(row}\OperatorTok{-}\DecValTok{1}\NormalTok{),] }\OperatorTok{+}\StringTok{ }\NormalTok{errors[row, ]}
\NormalTok{  \}}
\NormalTok{  simdata}
\NormalTok{\}}
\end{Highlighting}
\end{Shaded}

\subsubsection{C++}\label{c-3}

\begin{Shaded}
\begin{Highlighting}[]
\KeywordTok{library}\NormalTok{(Rcpp)}
\KeywordTok{library}\NormalTok{(RcppArmadillo)}
\KeywordTok{suppressMessages}\NormalTok{(}\KeywordTok{library}\NormalTok{(inline))}

\NormalTok{code <-}\StringTok{ '}
\StringTok{  arma::mat coeff = Rcpp::as<arma::mat>(a);}
\StringTok{  arma::mat errors = Rcpp::as<arma::mat>(u);}
\StringTok{  int m = errors.n_rows;}
\StringTok{  int n = errors.n_cols;}
\StringTok{  arma::mat simdata(m,n);}
\StringTok{  simdata.row(0) = arma::zeros<arma::mat>(1,n);}
\StringTok{  for (int row=1; row<m; row++) \{}
\StringTok{    simdata.row(row) = simdata.row(row-1) * trans(coeff) + errors.row(row);}
\StringTok{  \}}
\StringTok{  return Rcpp::wrap(simdata);}
\StringTok{'}
\NormalTok{Cppsim <-}\StringTok{ }\KeywordTok{cxxfunction}\NormalTok{(}\KeywordTok{signature}\NormalTok{(}\DataTypeTok{a=}\StringTok{"numeric"}\NormalTok{, }\DataTypeTok{u=}\StringTok{"numeric"}\NormalTok{), code, }\DataTypeTok{plugin=}\StringTok{"RcppArmadillo"}\NormalTok{);}
\end{Highlighting}
\end{Shaded}

\newpage

\subsubsection{Benchmarking}\label{benchmarking-3}

\begin{verbatim}
## Unit: microseconds
##          expr      min        lq      mean    median        uq       max
##    Rsim(A, u) 6945.951 7488.2425 8542.6906 7968.2435 8920.4275 85507.243
##  Cppsim(A, u)  206.612  222.2035  252.8797  232.3385  249.8865  1044.693
##  neval
##   1000
##   1000
\end{verbatim}

\includegraphics{main_script_files/figure-latex/unnamed-chunk-16-1.pdf}

\newpage

\subsubsection{Summary}\label{summary}

\begin{enumerate}
\def\labelenumi{\arabic{enumi}.}
\item
  Compile and link using either \texttt{inline} or \texttt{Rcpp} or
  both. You may use \texttt{R\ CMD\ SHLIB} to create a shared object,
  but you would need to load the shared object into \texttt{R} yourself.
\item
  Don't try to use \texttt{C++} for things that are easily vectorized in
  R.
\end{enumerate}

\newpage

\begin{center}
\LARGE
Thank You.
\end{center}


\end{document}
